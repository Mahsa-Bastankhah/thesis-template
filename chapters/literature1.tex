
\فصل{کارهای پیشین}
\قسمت{اهمیت متعادل نگه داشتن کانال های شبکه پرداخت}
به صورت سنتی مهم ترین الگوریتم های مسیریابی برای تراکنش ها در \کد{Lightning Network} همچون \کد{Flare}
~\cite{Prihodko2016FlareA}،
 \کد{SilentWhispers}
~\cite{MalavoltaMKM17} و
\کد{SpeedyMurmurs}
~\cite{RoosMKG18}.
بیشتر روی افزایش نرخ تراکنش بر ثانیه تمرکز میکنند تا متعادل نگه داشتن کانال ها.
اما اخیرا چندین پژوهش جدید در راستای بهبود تعادل کانال ها و اصلاح کانال های نامتعادل انجام شده است. مثلا
~\cite{KhalilG17}
برای اولین بار یک استراتژی امن برای متعادل کردن کانال های \کد{Lightning Network} پیشنهاد داد؛
~\cite{SivaramanVRNYMF20}
ارسال تراکنش ها در چندین مسیر را پیشنهاد میدهد که در حین اینکه نرخ ارسال تراکنش بالا میرود، به متعادل نگه داشتن کانال ها هم کمک میکند.
~\cite{Engelshoven2021TheMA}
توابع کارمزد را به نحوی تغییر میدهد که به کاربران انگیزه دهد که کانال ها را بیشتر در مسیری که به متعادل ماندن کانال کمک میکند استفاده کنند؛ و 
~\cite{LiMZ20}
با پیش بینی تراکنش ها در آینده، سعی میکند برای میزان پولی که باید در لحظه ساخت به کانال اضافه شود تخمین مناسبی بزند.
اما در این پایان نامه ما مساله متعادل نگه داشتن کانال ها را از نگاه دیگری بررسی میکنیم. ما هزینه های پردازش تراکنش ها را در نظر میگیریم و الگوریتمی طراحی میکنیم که این هزینه را کمینه کند. از آنجاییکه کانال های نامتعادل معمولا مجبور به رد کردن تعداد زیادی تراکنش میشوند، هزینه این کانال ها بالاست و در نتیجه الگوریتمی که ما طراحی میکنیم به صورت غیرمستقیم از کانال های نامتعادل و یکسو شده اجتناب میکند. تفاوت دیگری که این پایان نامه با سایر پروژه های موجود در این حوزه دارد این است که ما هیچ فرض خاصی روی توزیع دنباله تراکنش های آینده نمیکنیم.

\قسمت{متعادل کردن کانال برون بلاکچینی}\footnote{Off-chain rebalancing}
متعادل کردن برون بلاکچینی به عنوان جایگزینی ارزان (به جای بستن کانال و باز کردن کانال جدید) برای اضافه کردن پول به یک سمت کانال مورد بررسی تحقیقات گذشته قرار گرفته است.
برای نسخه های مختلف \کد{Lightning Network} همچون \کد{c-lightning} 
\footnote{https://github.com/lightningd/plugins/tree/master/rebalance} 
و \کد{ lnd}
\footnote{https://github.com/bitromortac/lndmanage}
 پلاگین هایی \footnote{plugin}
پیاده سازی شده است که از متعادل کردن برون بلاکچینی پشتیبانی میکند.
~\cite{PickhardtN20}
یک روش ابتکاری ارائه میدهد که به طور اتوماتیک زمان هایی که متعادل کردن برون بلاکچینی باید انجام شود را تشخیص میدهد.
ما هم در این پایان نامه به بررسی زمان هایی که متعادل کردن برون بلاکچینی انجام شود میپردازیم اما این تصمیم را در کنار و وابسته به سایر تصمیمات مدیریتی کانال همچون پذیرفتن یا رد کردن تراکنش ها میگیریم.
اخیرا برخی از مقاله ها مانند
~\cite{hidenseek} و ~\cite{KhalilG17}
مساله متعادل کردن را در کل کانال های \کد{Lightning Network} به صورت یک مساله جهانی\footnote{global}
در نظر میگیرند و این مساله را به فرم یک مساله
\کد{LP}
در می آورند و جواب بهینه را بدین صورت پیدا میکنند.
این دسته از مقالات به طور مستقیم به روشی که ما در این پایان نامه در پی گرفته ایم مرتبط نیستند زیرا ما در اینجا بر یک تک کانال متمرکز میشویم نه بر کل شبکه کانال های پرداخت.

\قسمت{استفاده از الگوریتم های آنلاین برای حل مسائل شبکه کانال های پرداخت}
~\cite{AvarikiotiBWW2019} و ~\cite{FazliNS2021}
نزدیک ترین کارها به این پایان نامه هستند زیرا آن ها هم از الگوریتم های آنلاین (آنالیز بدترین حالت) برای تصمیم گیری درباره پذیرش تراکنش ها در شبکه کانال های پرداخت استفاده میکنند.
~\cite{AvarikiotiBWW2019}
مسئله پذیرش یا عدم پذیرش تراکنش ها در یک تک کانال را در نظر میگیرد اما بر خلاف ما، امکان شارژ کردن درون بلاکچینی یا متعادل کردن برون بلاکچینی را به کاربران نمیدهد. 
~\cite{AvarikiotiBWW2019} 
اثبات میکند که برای مدلی که در نظر گرفته الگوریتم آنلاینی با ضریب رقابتی محدود نمیتوان پیدا کرد.

~\cite{FazliNS2021}
مسئله زمان بندی شارژ کردن درون بلاکچینی را در نظر میگیرد اما اجازه ی رد کردن تراکنش یا انجام متعادل کردن برون بلاکچینی را به کاربران نمیدهد.


\قسمت{ارتباط مسئله ما با مسائل مشابه در شبکه های مخابراتی}
مسائل کنترل پذیرش
\footnote{Admission Control}
به طور مثال \کد{online call admission} سالهاست که در شبکه های مخابراتی مورد بررسی قرار گرفته است
~\cite{aspnes1997line,lukovszki2015online}.
اما مسائلی که در شبکه های مخابراتی بررسی میشود دو تفاوت عمده با مسئله کنترل پذیرش در شبکه کانال های پرداخت دارد:\\
1) در شبکه های مخابراتی، ظرفیت یک لینک در یک جهت مستقل از جریان اطلاعاتی که از سمت دیگر لینک وارد میشود است در حالیکه در مسئله شبکه کانال های پرداخت، آمدن تراکنش ها از سمت دیگر کانال باعث افزایش موجودی سمت پذیرنده میشود و پذیرنده میتواند بعدا از این پول برای پذیرش تراکنش ها استفاده کند.\\
2) انتقال اطلاعات در یک لینک باعث کاهش ظرفیت لینک در آینده نمیشود و فقط از ظرفیت موجود لحظه ای لینک میکاهد، در صورتی که در شبکه کانال های پرداخت پذیرفتن یک تراکنش از یک سمت به سمت دیگر کانال باعث جابجایی همیشگی پول میشود.

این دو تفاوت عمده باعث میشود که مسئله ی کنترل پذیرش در شبکه کانال های پرداخت از نظر الگوریتمی مسئله ای کاملا متفاوت با پذیرش در شبکه های مخابراتی باشد.





