

\فصل{برخی نکات نگارشی}

این فصل حاوی برخی نکات ابتدایی ولی بسیار مهم در نگارش متون فارسی است. 
نکات گردآوری‌شده در این فصل به‌ هیچ‌ وجه کامل نیست، 
ولی دربردارنده‌ی حداقل مواردی است که رعایت آن‌ها در نگارش پایان‌نامه ضروری به نظر می‌رسد.

\قسمت{فاصله‌گذاری}

\شروع{شمارش}

\فقره 
علائم سجاوندی مانند نقطه، ویرگول، دونقطه، نقطه‌ویرگول، علامت سؤال، و علامت تعجب (. ، : ؛ ؟ !) بدون فاصله از کلمه‌ی پیشین خود نوشته می‌شوند، ولی بعد از آن‌ها باید یک فاصله‌ قرار گیرد. مانند: من، تو، او.
\فقره 
علامت‌های پرانتز، آکولاد، کروشه، نقل قول و نظایر آن‌ها بدون فاصله با عبارات داخل خود نوشته می‌شوند، ولی با عبارات اطراف خود یک فاصله دارند. مانند: (این عبارت) یا {آن عبارت}.
\فقره 
دو کلمه‌ی متوالی در یک جمله همواره با یک فاصله از هم جدا می‌شوند، ولی اجزای یک کلمه‌ی مرکب باید با نیم‌فاصله\زیرنویس{«نیم‌فاصله» فاصله‌‌ای مجازی است که در عین جدا کردن اجزای یک کلمه‌ی مرکب از یک‌دیگر، آن‌ها را نزدیک به هم نگه می‌دارد. معمولاً برای تولید این نوع فاصله در صفحه‌کلید‌های استاندارد از ترکیب Shift+Space استفاده می‌شود.}‌‌
 از هم جدا شوند. مانند: کلاسِ درس، محبت‌آمیز، دوبخشی.
\پایان{شمارش}


\قسمت{شکل حروف}

\شروع{شمارش}

\فقره 
در متون فارسی به جای حروف «ك» و «ي» عربی باید از حروف «ک» و «ی» فارسی استفاده شود. همچنین به جای اعداد عربی مانند ٥ و ٦ باید از اعداد فارسی مانند ۵ و ۶ استفاده نمود. 
برای این کار، توصیه می‌شود صفحه‌کلید‌ فارسی استاندارد\زیرنویس{\href{http://persian-computing.ir/download/Iranian_Standard_Persian_Keyboard_(ISIRI_9147)_(Version_2.0).zip}{صفحه‌کلید فارسی استاندارد برای ویندوز}، تهیه‌شده توسط بهنام اسفهبد} را بر روی سیستم خود نصب کنید.
\فقره 
عبارات نقل‌قول‌شده یا مؤکد باید درون علامت نقل قولِ «» قرار گیرند، نه ''``. مانند: «کشور ایران».
\فقره 
کسره‌ی اضافه‌ی بعد از «ه» غیرملفوظ به صورت «ه‌ی» نوشته می‌شود، نه «هٔ». مانند: خانه‌ی علی، دنباله‌ی فیبوناچی.

        تبصره‌: اگر «ه» ملفوظ باشد، نیاز به «‌ی» ندارد. مانند: فرمانده دلیر، پادشه خوبان. 

\فقره 
پایه‌های همزه در کلمات، همیشه «ئـ» است، مانند: مسئله و مسئول، مگر در مواردی که همزه ساکن است که در این ‌صورت باید متناسب با اعراب حرف پیش از خود نوشته شود. مانند: رأس، مؤمن. 

\پایان{شمارش}


\قسمت{جدانویسی}

\شروع{شمارش}

\فقره 
اجزای فعل‌های مرکب با فاصله از یک‌دیگر نوشته می‌شوند، مانند: تحریر کردن، به سر آمدن.
\فقره 
علامت استمرار، «می»، توسط نیم‌فاصله از جزء‌ بعدی فعل جدا می‌شود. مانند: می‌رود، می‌توانیم.
\فقره 
شناسه‌های «ام»، «ای»، «ایم»، «اید» و «اند» توسط نیم‌فاصله، و شناسه‌ی «است» توسط فاصله از کلمه‌ی پیش از خود جدا می‌شوند. مانند: گفته‌ام، گفته‌ای، گفته است.
\فقره 
علامت جمع «ها» توسط نیم‌فاصله از کلمه‌ی پیش از خود جدا می‌شود. مانند: این‌ها، کتاب‌ها.
\فقره 
«به» همیشه جدا از کلمه‌ی بعد از خود نوشته می‌شود، مانند: به‌ نام و به آن‌ها، مگر در مواردی که «بـ» صفت یا فعل ساخته است. مانند: بسزا، ببینم.
\فقره 
«به» همواره با فاصله از کلمه‌ی بعد از خود نوشته می‌شود، مگر در مواردی که «به» جزئی از یک اسم یا صفت مرکب است. مانند: تناظر یک‌به‌یک، سفر به تاریخ. 
\پایان{شمارش}


\قسمت{جدانویسی مرجح}

\شروع{شمارش}

\فقره 
اجزای اسم‌ها، صفت‌ها، و قیدهای مرکب توسط نیم‌فاصله از یک‌دیگر جدا می‌شوند. مانند: دانش‌جو، کتاب‌خانه، گفت‌وگو، آن‌گاه، دل‌پذیر.

        تبصره: اجزای منتهی به «هاء ملفوظ» را می‌توان از این قانون مستثنی کرد. مانند: راهنما، رهبر. 

\فقره 
علامت صفت برتری، «تر»، و علامت صفت برترین، «ترین»، توسط نیم‌فاصله از کلمه‌ی پیش از خود جدا می‌شوند. مانند: بیش‌تر، کم‌ترین.

        تبصره‌: کلمات «بهتر» و «بهترین» را می‌توان از این قاعده مستثنی نمود. 

\فقره 
پیشوندها و پسوندهای جامد، چسبیده به کلمه‌ی پیش یا پس از خود نوشته می‌شوند. مانند: همسر، دانشکده، دانشگاه.

        تبصره‌: در مواردی که خواندن کلمه دچار اشکال می‌شود، می‌توان پسوند یا پیشوند را جدا کرد. مانند: هم‌میهن، هم‌ارزی. 

\فقره 
ضمیرهای متصل چسبیده به کلمه‌ی پیش‌ از خود نوشته می‌شوند. مانند: کتابم، نامت، کلامشان. 

\پایان{شمارش}


\شروع{مسئله}%[دورهای پوشای کمینه]
 گراف غیر جهت‌دار  \lr{$G=(V,E)$} به همراه $m$ رأس مشخص $d_1, d_2, \ldots, d_m$ از $V$ به عنوان انبار و $m$ تابع وزن $w_1, w_2, \ldots,  w_m: V \times V \rightarrow \IR^+$ داده شده است. در هر یک از انبارها یک عامل (وسیله‌ی نقلیه) قرار دارد. هدف یافتن $m$ دور است که از $d_1, d_2, \ldots,  d_m$ شروع شده و اجتماع آن‌ها تمام رأس‌های گراف را بپوشاند طوری که مجموع هزینه‌ی این دورها کمینه شود.
 هزینه‌ی دور $i$اُم با تابع $w_i$ اندازه‌گیری می‌شود.
 \پایان{مسئله}
 
در صورت همگن مسئله، هزینه‌ی پیمایش یال‌ها برای همه‌ی عوامل یکسان است و در گونه‌ی ناهمگن،
این هزینه برای عوامل مختلف می‌تواند متفاوت باشد. 
از آن‌ جایی که صورت ناهمگن مسئله کم‌تر مورد توجه قرار گرفته است،
در این تحقیق سعی شده است که تمرکز بر روی این گونه از مسئله باشد.
همچنین علاوه بر دورهای ناهمگن، درخت‌ها و مسیرهای ناهمگن نیز در این پایان‌نامه مورد بررسی قرار خواهند گرفت.


مسئله‌ی مسیریابی وسایل نقلیه کاربردهای بسیار گسترده‌ای در حوزه‌ی حمل و نقل دارد. برای نخستین بار این مسئله برای مسیریابی تانکرهای سوخت‌رسان مطرح شد~\cite{Dantzig}. اما امروزه با پیشرفت‌های گسترده‌ای که در زمینه‌ی تکنولوژی روی داده است از راه‌حل‌های این مسئله در امور روزمره از جمله سیستم توزیع محصولات، تحویل نامه، جمع‌آوری زباله‌های خانگی و غیره استفاده می‌شود. در نظر گرفتن فرض ناهمگن بودن هم با توجه به اینکه معمولاً عوامل توزیع در یک سیستم، یکسان نیستند و تفاوت‌هایی در میزان مصرف سوخت و غیره دارند، راه‌حل‌های مناسب‌تری برای مسائل این حوزه می‌تواند ارائه دهد.
گونه‌های مختلفی از مسائل مسیریابی وسایل نقلیه در \مرجع{formulation1,formulation2,formulation3}
بیان شده است.




همان‌طور که ذکر شد مسئله‌ی مسیریابی وسایل نقلیه‌ی ناهمگن صورت عمومی مسئله‌ی فروشنده دوره‌گرد می‌باشد. 
مسئله‌ی فروشنده‌ی دوره‌گرد در حوزه‌ی مسائل ان‌پی-سخت\پاورقی{NP-hard} قرار می‌گیرد و با فرض $P \neq NP$ الگوریتم دقیق با زمان چندجمله‌ای برای آن وجود ندارد. بنابراین برای حل کارای این مسائل از الگوریتم‌های تقریبی\پاورقی{Approximation Algorithm}  استفاده می‌شود.

مسئله‌ی فروشنده‌ی دوره‌گرد در حالتی که تنها یک فروشنده در گراف حضور داشته باشد، دو الگوریتم تقریبی معروف دارد.
در الگوریتم اول با دو برابر کردن درخت پوشای کمینه\پاورقی{Minimum Spanning Tree} و میانبر کردن\پاورقی{Shortcut} دورهای بدست آمده، الگوریتمی با ضریب تقریب 2 ارائه می‌شود.
در الگوریتم دوم که متعلق به کریستوفایدز\پاورقی{Christofides}~\cite{Christofides} است، به کمک ساخت دور اویلری\پاورقی{Eulerian Cycle} بر روی اجتماع یال‌های درخت پوشای کمینه و یال‌های تطابق کامل کمینه\پاورقی{Minimum Perfect Matching} از گره‌های درجه‌ی فرد همان درخت، و میانبر کردن این دور، ضریب تقریب $1.5$  ارائه می‌شود.
با گذشت حدود ۴۰ سال از ارائه‌ی این الگوریتم، تا کنون 
ضریب تقریب بهتری برای این مسئله پیدا نشده است.

اخیراً با بهره‌گیری از روش کریستوفایدز و بسط آن برای مسئله‌ی فروشنده‌ی دوره‌گرد چندگانه‌ی همگن (در این حالت از مسئله تعداد فروشنده‌ها در گراف بیش از یکی است و هزینه‌ی پیمایش یال‌ها برای همه‌ی عوامل یکسان است) ضریب تقریب $1.5$ ارائه شده است~\cite{Xu}. در روش مطرح شده بعد از به دست آوردن درخت‌های پوشای کمینه برای هر انبار، به جای استفاده از روش دو برابر کردن یال‌ها، روش کریستوفایدز اعمال می‌شود. به راحتی می‌توان نشان داد که صرف اعمال الگوریتم کریستوفایدز به هر یک از درخت‌های بدست آمده، ضریب تقریب  $1.5$ را بدست نمی‌دهد. بنابراین در روش مذکور، الگوریتم کریستوفایدز روی کل جنگل بدست آمده اعمال می‌شود. نشان داده شده است که با استفاده از یک سیاست جایگزینی مناسب بین یال‌هایی که در جنگل کمینه، موجود هستند و آن‌هایی که در این مجموعه حضور ندارند و اعمال کریستوفایدز روی این جنگل‌ها، می‌توان جوابی تولید کرد که بدتر از $1.5$ برابر جواب بهینه نباشد.


همان‌طور که گفته شد نسخه‌ی ناهمگن این مسئله کمتر مورد توجه قرار گرفته است. در گونه‌ی ناهمگن، بیش از یک عامل (فروشنده) در اختیار داریم که در شروع، هر یک از آن‌ها در گره‌های مجزایی که با عنوان انبار معرفی می‌شوند قرار دارند و هزینه‌ی پیمایش یال‌ها برای هریک از عوامل می‌تواند متفاوت از سایر عامل‌ها باشد. در صورتی که تعداد انبارها $m$ فرض شود از جمله کارهای انجام شده در این مورد ارائه ضریب تقریب $4m$ به کمک حل برنامه‌ریزی خطی تعدیل شده\پاورقی{Linear Programming Relaxation}  و ساخت درخت پوشای کمینه~\cite{4m}، ضریب تقریب $1.5m$ به کمک حل تعدیل برنامه‌ریزی خطی با روش بیضی\پاورقی{Ellipsoid Method} و اعمال الگوریتم کریستوفایدز~\cite{1.5m} و ضریب تقریب 2 به کمک راه حل اولیه-دوگان\پاورقی{Primal-Dual} می‌باشد، روش اولیه-دوگان تنها برای حالتی که دو عامل وجود دارد و هزینه‌ی پیمایش یال‌ها برای یک عامل بیشتر از عامل دیگر باشد مطرح شده است~\cite{Primal_Dual}.






در برنامه‌ریزی‌ ریاضی سعی بر بهینه‌سازی (کمینه یا بیشینه کردن) یک
 تابع هدف با توجه به تعدادی محدودیت است. شکل خاصی از این برنامه‌ریزی که توجه ويژه‌ای به آن در علوم کامپیوتر شده است برنامه‌ریزی خطی می‌باشد. در برنامه‌ریزی خطی به دنبال بهینه کردن یک تابع هدف خطی با توجه به تعدادی محدودیت خطی می‌باشیم. شکل استاندارد یک برنامه‌ریزی خطی به صورت زیر است.
\begin{alignat}{3}
\text{\lr{minimize}}   & \quad &&  c^T x       \label{LP-def}  \\
\text{\lr{s.t.}}           & &&  Ax \geq b   \notag           \\
                       	& &&   x \geq 0     \notag 
\end{alignat}

در روابط فوق، $x$ بردار متغیرها،  $b, c$ بردارهای ثابت و $A$ ماتریس ضرایب می‌باشد. به سادگی قابل مشاهده است که رابطه‌ی~(\ref{LP-def}) می‌تواند شکل‌های مختلفی از برنامه‌ریزی خطی را در بر بگیرد. به طور خاص اگر روابط قید‌ها به حالت $(A^\prime x=b^ \prime)$ یا در جهت برعکس $(A^{\prime\prime} x \leq b^{\prime\prime} )$ باشد یا تابع هدف به صورت بیشینه‌سازی باشد. همه‌ی این موارد با تغییر کمی در رابطه‌ی~(\ref{LP-def}) یا اضافه کردن پارامتر و متغیر جدید قابل مدل کردن می‌باشد. برای مطالعه‌ی بیشتر در مورد برنامه‌ریزی خطی می‌توانید به~\cite{Sch86}  مراجعه کنید.

هر برنامه‌ریزی خطی مطرح شده به شکل بالا قابل حل در زمان چندجمله‌ای است~\cite{Kha79,Kar84}. روش بیضوی~\cite{Kha79} از این مزیت بهره می‌برد که نیازی به بررسی همه‌ی محدودیت‌ها ندارد. در حقیقت این روش با در اختیار داشتن یک دانای کل جداکننده\پاورقی{Separation Oracle} می‌تواند جواب بهینه‌ی برنامه‌ریزی خطی را در زمان چندجمله‌ای بدست آورد. دانای کل جداکننده رویه‌ای است که با گرفتن بردار $x$ به عنوان ورودی مشخص می‌کند که آیا $x$ همه‌ی محدودیت‌های برنامه‌ریزی خطی را برآورده می‌سازد یا خیر‌، در حالت دوم دانای کل جداکننده حداقل یک محدودیت نقض شده را گزارش می‌دهد. این مسئله زمانی کمک کننده خواهد بود که برنامه‌ریزی خطی دارای تعداد نمایی محدودیت باشد اما ساختار ترکیبیاتی محدودیت‌ها امکان ارزیابی امکان‌پذیر بودن جواب مورد نظر را فراهم آورد.

برای هر برنامه‌ریزی خطی می‌توان شکل دوگان آن را نوشت. به برنامه‌ی اصلی، برنامه‌ی اولیه گفته  می شود. دوگان رابطه‌ی~(\ref{LP-def}) به صورت زیر می‌باشد:
\begin{alignat}{3}
\text{\lr{maximize}}   & \quad &&    b^T y           \label{DUAL-def}  \\
\text{\lr{s.t.}}          &  &&    A^T y \leq c  \notag  \\
                       &  &&	y \geq 0        \notag 
\end{alignat}

برنامه‌های اولیه و دوگان به کمک قضایای دوگانی زیر با هم ارتباط دارند.


\شروع{قضیه}[قضیه‌ی دوگانی ضعیف] 
یک برنامه‌ریزی خطی کمینه‌سازی با تابع هدف $c^T x$ و صورت دوگان آن با تابع هدف $b^T y$ را در نظر بگیرید. برای هر جواب ممکن $x$ برای برنامه‌ی اولیه و جواب ممکن $y$ برای برنامه‌ی دوگان، رابطه‌ی  $b^T y \leq c^T x$ برقرار است.
\پایان{قضیه}

درستی قضیه‌ی بالا به راحتی قابل تصدیق است زیرا $b^T y \leq (Ax)^T y = x^T A^T y \leq x^T c = c^T x$، برقراری نامساوی‌ها از نامساوی‌های برنامه‌‌ی اولیه و دوگان حاصل می‌شود. قضیه‌ی قوی دوگانی در~\cite{N47} به صورت زیر بیان شده است.


\شروع{قضیه}[قضیه‌ی دوگانی قوی] 
یک برنامه‌ریزی خطی کمینه‌سازی با تابع هدف $c^T x$ و صورت دوگان آن با تابع هدف $b^T y$ را در نظر بگیرید. اگر برنامه‌ی اولیه یا دوگان دارای جواب بهینه‌ی نامحدود باشد، برنامه‌ی متقابل فاقد جواب ممکن است. در غیر این صورت مقدار بهینه‌ی توابع هدف دو برنامه مساوی خواهد بود، به عبارت دیگر جواب $x^*$ برای برنامه ی اولیه و جواب $y^*$  برای برنامه‌ی دوگان وجود خواهد داشت که $c^T x^* = b^T y^*$.
\پایان{قضیه}
 
درصورتی مقادیر متغیر‌ها محدود به اعداد صحیح شود به عنوان مثال $x \in \{0, 1\}^n$ به این شکل از برنامه‌ریزی،  برنامه‌ریزی صحیح می‌گوییم. این شکل از برنامه‌ریزی به سادگی قابل بهینه‌سازی نیستند. برداشتن محدودیت صحیح بودن متغیرها، برنامه‌ریزی خطی تعدیل‌شده را نتیجه می‌دهد. بهترین الگوریتم‌ها برای بسیاری از مسائل با گرد کردن جواب برنامه‌ریزی خطی تعدیل شده به مقادیر صحیح یا با بهره‌گیری از ویژگی‌های برنامه‌ریزی خطی 
(نظیر روش اولیه-دوگان~\cite{ISAAC12}) حاصل شده است. 
دقت کنید که جواب برنامه‌ریزی خطی تعدیل‌شده برای یک مسئله، به عنوان حد پایینی برای جواب بهینه‌ی آن مسئله محسوب می‌گردد.

زمانی که از برنامه‌ریزی خطی تعدیل شده برای حل یا تقریب زدن یک مسئله استفاده می‌شود، گپ صحیح\پاورقی{Integrality Gap} برنامه‌ریزی خطی معمولاً بیانگر این است که جواب ما تا چه حد می‌تواند مناسب باشد. برای یک مسئله‌ی کمینه‌سازی، گپ صحیح به صورت 
کوچک‌ترین کران بالای مقدار برنامه‌ریزی خطی تعدیل شده برای نمونه‌ی $I$ تقسیم بر مقدار بهینه‌ برای نمونه‌ی $I$
تعریف می‌شود.
گپ صحیح برای مسئله‌ی بیشینه‌سازی به صورت معکوس تقسیم مطرح شده بیان می‌گردد.

بسیاری از مسائل بهینه‌سازی مهم و پایه‌ای
ان‌پی-سخت هستند. بنابراین، با فرض $P \neq NP$
نمی‌توان الگوریتم‌هایی با زمان چندجمله‌ای برای این مسائل ارائه کرد.
روش‌های متداول برای برخورد با این مسائل عبارت‌اند از:

\شروع{فقرات}
\فقره مسئله را فقط برای حالات خاص حل نمود.
\فقره با استفاده از روش‌های جست‌وجوی تمام حالات، 
مسئله را در زمان غیرچندجمله‌ای حل نمود.
\فقره در زمان چندجمله‌ای، تقریبی از جواب بهینه را به دست آورد.
\پایان{فقرات}

در این پایان‌نامه تمرکز بر روی روش سوم یعنی
استفاده از الگوریتم‌های تقریبی است.
الگوریتم‌های تقریبی قادرند جوابی نزدیک به جواب بهینه 
را در زمان چندجمله‌ای پیدا کنند.

مسئله‌ی بهینه‌سازی (کمینه‌سازی یا بیشینه‌سازی) $P$ را در نظر بگیرید. 
فرض کنید هر نمونه از مسئله‌ی $P$  دارای یک مجموعه‌ی ناتهی 
از جواب‌های ممکن\پاورقی{feasible} است. به هر جواب ممکن،
یک عدد مثبت به عنوان هزینه (یا وزن) آن نسبت داده شده است. 
مسئله‌ی $P$ با شرایط فوق یک مسئله‌ی 
\موکد{ان‌پی-بهینه‌سازی} (\lr{NP-Optimization}) است،


به ازای هر نمونه‌ی $I$ از یک مسئله‌ی ان‌پی-بهینه‌سازی $P$،
هزینه‌ی جواب بهینه برای $I$ را با $\OPT(I)$ نشان می‌دهیم.
همچنین، هزینه‌ی جواب تولیدشده توسط الگوریتم تقریبی 
بر روی $I$ را با  $\ALG(I)$ نشان می‌دهیم.

\حذف{
\شروع{تعریف}
یک الگوریتم تقریبی برای مسئله‌ی $P$ 
دارای \موکد{ضریب تقریب افزایشی}\پاورقی{additive approximation factor} $c$ است 
اگر برای هر نمونه‌ی $I$ از~$P$:

\[
	|ALG(I) - OPT(I)| \leq c. 
\]
\پایان{تعریف}
} %پایان حذف

\شروع{تعریف}
یک الگوریتم تقریبی برای مسئله‌ی $P$ دارای \موکد{ضریب تقریب} $\alpha$ است 
اگر برای هر نمونه‌ی $I$ از~$P$:
\[
	\max \left\{ \frac{ALG(I)}{OPT(I)} , \frac{OPT(I)}{ALG(I)} \right\} \leq \alpha. 
\]
\پایان{تعریف}


یک الگوریتم تقریبی با ضریب تقریب $\alpha$،
یک \موکد{الگوریتم $\alpha$-تقریبی} نامیده می‌شود.
نمونه‌هایی از ضرایب تقریب متداول برای مسائل بهینه‌سازی 
در جدول~\رجوع{جدول:ضرایب‌تقریب} آمده است.


\begin{table}[t]
\centering
\begin{latin}
\begin{tabular}{|c|c|}
\hline
\rl{ضریب تقریب} & \rl{مسئله‌}
\\
\hline
\hline
$1+\eps$ $(\eps > 0)$ & Euclidian TSP \\
const $c$ & Vertex Cover \\
$\log n$  & Set Cover \\
$n^\delta$ $(\delta <1)$ &  Coloring \\
$\infty$  & TSP \\
\hline
\end{tabular}
\end{latin}
\شرح{نمونه‌هایی از ضرایب تقریب برای مسائل بهینه‌سازی}
\برچسب{جدول:ضرایب‌تقریب}
\end{table}


%----------------------------- مقدمه ----------------------------------



به عنوان اولین مسئله از مجموعه مسائل بهینه‌سازی،
در این بخش به بررسی مسئله‌ی پوشش رأسی می‌پرازیم.
این مسئله به صورت زیر تعریف می‌شود.

\شروع{مسئله}[پوشش رأسی]
گراف $G=(V,E)$  و تابع هزینه‌ی $w:V \rightarrow \IR^{+}$ داده شده است.
زیرمجموعه‌ی $C \subseteq V$ با حداقل هزینه را بیابید طوری که 
به ازای هر یال $uv \in E$، حداقل یکی ازدو رأس $u$ و $v$  در مجموعه‌ی $C$ باشد.
\پایان{مسئله}

%مسئله‌ی پوشش رأسی در حالت غیروزن‌دار، یعنی وقتی همه‌ی رأس‌ها دارای وزن واحد هستند، 
%مسئله‌ی \موکد{پوشش رأسی اندازه‌ای}\پاورقی{\lr{Cardinality Vertex Cover}} نامیده می‌شود. 

\شروع{شکل}
[t]\centerfig{cover.tex}{1}

\شرح{گراف $G$ و یک پوشش رأسی برای آن}
\برچسب{شکل:پوشش رأسی کاردینال}
\پایان{شکل}

شکل~\رجوع{شکل:پوشش رأسی کاردینال} 
نمونه‌ای از یک پوشش رأسی را نشان می‌دهد.
در زیر یک الگوریتم حریصانه برای مسئله‌ی پوشش رأسی غیروزن‌دار ارائه شده است.


\شروع{الگوریتم}{پوشش رأسی حریصانه}
\دستور{قرار بده $C = \emptyset$}
\تاوقتی{$E$ تهی نیست}
%\اگر{$|E| > 0$}
%	\دستور{یک کاری انجام بده}
%\پایان‌اگر
\دستور{یال دل‌‌خواه $uv \in E$ را انتخاب کن}
\دستور{$C \leftarrow C \cup \{ u,v \}$}
\دستور{تمام یال‌های واقع بر $u$ یا $v$ را از $E$ حذف کن}
\پایان‌تاوقتی
\دستور {$C$ را برگردان}
\پایان{الگوریتم}


به سادگی می‌توان مشاهده نمود که خروجی الگوریتم~\رجوع{الگوریتم: پوشش رأسی حریصانه}
یک پوشش رأسی است.
در ادامه نشان خواهیم داد که اندازه‌ی پوشش رأسی تولیدشده توسط الگوریتم
حداکثر دو برابر اندازه‌ی پوشش رأسی کمینه است.

\شروع{قضیه} \برچسب{قضیه:پوشش رأسی}
$\OPT \leq |C| \leq 2 \OPT$.
\پایان{قضیه}

\شروع{اثبات}
از آن جایی که $C$ یک پوشش رأسی است، نامساوی سمت چپ بدیهی است.
فرض کنبد $M$ مجموعه‌ی تمام یال‌هایی باشد که توسط الگوریتم انتخاب شده‌اند. 
از آن‌ جایی که هیچ دو یالی در $M$ دارای رأس مشترک نیستند، 
هر پوشش رأسی (از جمله پوشش رأسی بهینه) 
باید حداقل یک رأس از هر یال موجود در $M$ را بپوشاند. بنابراین
$$|M| \leq \OPT.$$
از طرفی می‌دانیم $|C| = 2|M|$. در نتیجه
$$
	|C| = 2|M| \leq 2 \OPT.
$$
\پایان{اثبات}

بنا بر قضیه‌ی~\رجوع{قضیه:پوشش رأسی}، 
الگوریتم~\رجوع{الگوریتم: پوشش رأسی حریصانه} یک الگوریتم ۲-تقریبی است.
مثال زیر نشان می‌دهد که ضریب تقریب~۲ برای این الگوریتم محکم است.
گراف دو بخشی کامل $K_{n,n}$ را در نظر بگیرید.
پوشش رأسی تولیدشده توسط الگوریتم حریصانه بر روی این گراف
شامل تمامی $2n$ رأس گراف خواهد بود، در صورتی که پوشش رأسی بهینه
شامل نصف این تعداد، یعنی $n$ رأس است.


 


