
\فصل{مقدمه}


بلاکچین هایی همچون بیتکوین به دلیل ماهیت توزیع شده والگوریتم اجماع پیچیده و وقت گیری که دارند با مشکل عدم مقیاس پذیری روبرو هستند. به طور مثال بلاکچین بیتکوین تنها میتواند 7 تراکنش در ثانیه را پردازش کند در حالیکه رقبای متمرکز بلاکچین همچون \کد{visa} بیش از هزاران تراکنش را در هر ثانیه پردازش میکنند. به علاوه، تایید شدن تراکنش ها در بلاکچین ها معمولا حداقل چند دقیقه به طول می انجامد. یکی از مورد استقبال ترین راه حل هایی که برای حل مشکل مقیاس ناپذیری و کندی بلاکچین ارائه شده است استفاده از شبکه کانال های پرداخت\پاورقی{payment channel network} است.
شبکه کانال های پرداخت اولین بار با پیاده سازی \کد{Lightning Network} روی بلاکچین بیتکوین معرفی شد.
~\cite{poon2015lightning}
بعد ها شبکه کانال های پرداخت \کد{Raiden} هم با الهام از \کد{Lightning Network} بر بلاکچین اتریوم\پاورقی{Ethereum} توسعه داده شد.~\cite{raiden}
کاربران میتوانند با ارسال یک تراکنش ایجاد کانال \پاورقی{channel creation} روی بلاکچین، یک کانال پرداخت ایجاد کنند. با این تراکنش در واقع طرفین کانال پرداخت، مقداری پول را در این کانال پرداخت به سپرده میگذارند. پس از ایجاد کانال، طرفین میتوانند بدون مراجعه به بلاکچین و با رد و بدل کردن تعدادی امضای دیجیتال برای هم تراکنش محلی فوری\پاورقی{instant} با کارمزد بسیار اندک و بفرستند. مبادله امضاهای دیجیتال برای حفظ امنیت مالی طرفین الزامی است؛ در واقع در اینجا طرفین کانال پرداخت به هم اعتماد ندارند و پیش فرض اولیه این است که هر کدام از طرفین ممکن است تلاش در دزدیدن پول دیگری کند؛ بنابراین پروتکل با استفاده از اصول رمزنگاری به نحوی طراحی شده است که از این امر جلوگیری کند. به عنوان مثالی از نحوه کارکرد کانال پرداخت، دو کاربر \کد{َA} و \کد{َB} را در نظر بگیرید که هنگام ایجاد کانال به ترتیب 2 و 4 بیتکوین در کانال به سپرده میگذارند. فرض کنید \کد{َA} میخواهد 1 بیتکوین به \کد{َB} بدهد؛ این دو نفر امضاهای دیجیتالی مبادله میکنند که وضعیت کانال را از 
$2-4$ 
به 
$1-5$ 
تغییر میدهد اما این تغییر در این مرحله بر بلاکچین منعکس نمیشود. در نهایت وقتی طرفین تصمیم به بستن کانال گرفتند آخرین امضاهای مبادله شده را که نمایانگر موجودی نهایی طرفین است را روی بلاکچین قرار میدهند و هر کدام از طرفین سهم خودش را دریافت میکند و کانال بسته میشود. مثلا در مثال بالا ممکن است پس از انجام رشته ای از تراکنش ها موجودی طرفین $3-3$ باشد، پس هر کدام از \کد{َA} و \کد{B} 3 بیتکوین دریافت کرده و کانال پرداخت بسته میشود. بدین ترتیب با ارسال تنها دو تراکنش \کد{on-chain}(تراکنش های ایجاد و بستن کانال) میتوان تعداد زیادی تراکنش \کد{off-chain} ارسال کرد.

از اتصال کاربران مختلف با کانال های پرداخت یک شبکه از کانال های پرداخت ایجاد میشود که میتواند افرادی که کانال پرداخت مستقیم به هم ندارند را هم با یک یا تعدادی واسطه به هم متصل کند. مثلا 3 کاربر \کد{A-B-C} را در نظر بگیرید که \کد{A-B} و \کد{B-C} کانال پرداخت دارند. در این صورت \کد{َA} و \کد{َC} اگرچه کانال پرداخت مشترک ندارند اما میتوانند از \کد{َB} به عنوان واسطه استفاده کنند و برای هم تراکنش ارسال کنند؛ بدین صورت که \کد{َA} مقدار پول مورد نظر را برای \کد{B} می فرستد و \کد{َB} همان مقدار پول را برای \کد{َC} میفرستد. این دو تراکنش ها \کد{atomic} هستند که به این معنی است که یا هر دو آنها انجام میشوند و یا هر دو برگشت میخورند. معمولا فرد واسطه یعنی \کد{َB} مقداری کارمزد از  \کد{َA} میگیرد اما این کرامزد در برابر کارمزد های شبکه بیتکوین بسیار ناچیز است و صرفا نقش ایجاد انگیزه برای واسطه ها را دارد. البته گاهی نود های واسطه ممکن است برخی تراکنش ها را به دلایلی رد کنند. مثلا ممکن است اندازه تراکنش بیشتر از موجودی  آن نود واسطه در کانال باشد یا اینکه موجودی نود واسطه را در حد غیر قابل قبولی کاهش دهد و یا اینکه میزان کارمزد آن مطلوب نود واسطه نباشد. 

یکی از مشکلات بسیار مهم شبکه کانال های پرداخت این است که بعد از ایجاد کانال هیچ راهی برای افزودن سپرده به کانال وجود ندارد. مثلا در مثال بالا در کانال \کد{A-B} فرض کنید با شروع از سپرده اولیه
 $2-4$
 ،\کد{A} 2 تراکنش هر کدام به ارزش 1 بیتکوین برای \کد{B} میفرستد؛ موجودی آنها
$0-6$   
خواهد بود و حالا پس از این تا زمانی که \کد{B}  تراکنشی برای \کد{A} نفرستد، \کد{A} نمیتواند تراکنشی بفرستد زیرا موجودی اش صفر است. به کانال پرداختی که در آن موجودی یک نفر صفر(یا بسیار کم است) کانال نامتعادل \پاورقی{depleted channel} میگوییم. کانال های نامتعادل برای کاربران عادی مطلوب نیستند زیرا امکان ایجاد تراکنش از یک سمت کانال را به کل از بین میبرند اما کانال های نامتعادل به خصوص برای نود های واسطه یا سرویس دهنده \پاورقی{service provider} بسیار مشکل ساز هستند. نود های سرویس دهنده نود هایی هستند که به شبکه کانال های پرداخت ملحق میشوند و با ذخیره کردن مقدار چشم گیری سپرده، تعداد زیادی کانال با کاربران زیادی ایجاد میکنند تا تراکنش های آنها را مسیریابی کنند و در ازای آن کارمزد بگیرند. داشتن کانال های نامتعادل توانایی نود های واسطه را در انتقال تراکنش ها از یک جهت کاهش میدهد و برای کسب و کار آنها مشکل ایجاد میکند. \کد{Lightning network} دو راه حل را برای حل مشکل کانال های نامتعادل پیشنهاد میدهد:


\برچسب{شمارش:راه های مقابله با کانال نامتعادل}\شروع{شمارش}

\فقره 
 شارژ کردن درون بلاکچینی: در این روش طرفین کانال نامتعادل آن کانال را میبندند و کانال جدیدی با سپرده بیشتر باز میکنند. این عمل باعث ایجاد دو تراکنش درون بلاکچینی میشود. یک تراکنش برای بستن کانال قدیمی و یک تراکنش برای ایجاد کانال جدید.
\فقره 
متعادل کردن برون بلاکچینی: این روش بدون مراجعه به بلاکچین و صرفا با تعدادی تراکنش برون بلاکچینی توزیع سپرده ها را در کانال نامتعادل تغییر میدهد و به نسبت روش قبل ارزان تر است. این عمل به این صورت انجام میگیرد که طرفین کانال یک مسیر دوم از کانال های پرداخت بین خودشان پیدا میکنند و با جایجایی پول در یک حلقه فردی که همه پول را در کانال نامتعادل به دست دارد برای فرد دیگر اندکی پول میریزد و معادل آن پول را در یک کانال دیگر از آن نود دریافت میکند.
\پایان{شمارش}

 از آنجاییکه هر دو روش بالا هزینه بر هستند و محدودیت هایی را اعمال میکنند، تصمیم گیری بر سر اینکه چه زمانی کدام یک از آنها انجام گیرد تصمیم سختی میتواند باشد. همچنین توجه کنید که به عنوان یک کاربر یا یک نود واسطه، معمولا نود اطلاعات دقیقی از تراکنش های آینده ندارد و در نتیجه نود ها باید سیاست تصمیم گیری ای را اتخاذ کنند که پیش فرض خاصی درباره توزیع تراکنش ها ندارد بنابراین سیاست تصمیم گیری باید آنلاین باشد. 
 
 
هدف این پایان نامه این است که سیاست آنلاینی طراحی کند یک کانال پرداخت را در کلی ترین حالت ممکن در نظر 
میگیرد و به سوالات زیر که برای بیشینه کردن سود و کمینه کردن هزینه طرفین کانال مطرح میشود پاسخ میدهد:

\شروع{شمارش}

\فقره 
چه زمانی ایجاد یک کانال پرداخت نسبت به ارسال درون بلاکچینی تراکنش به مقرون به صرفه است؟
\فقره 
اگر تصمیم به ایجاد کانال پرداخت شد، طرفین چه مقدار سپرده باید در آن قرار دهند؟
\فقره 
اگر طرفین میخواهند نقش واسطه را ایفا کنند چه تراکنش هایی را باید بپذیرند و چه تراکنش هایی را نپذیرند؟
\فقره 
اگر کانال پرداخت نامتعادل شد، طرفین باید کدام یک از راه های  مقابله با کانال نامتعادل را اتخاذ کنند و سپرده کانال را چقدر باید تغییر دهند؟
\پایان{شمارش}


\قسمت{اهمیت موضوع}


هدف از طراحی شبکه کانال های پرداخت در واقع ایجاد بستری ارزان و سریع برای انجام تراکنش های کوچک و روزانه \پاورقی{micro payment} است. بهره بری کاربران از شبکه کانال های پرداخت تا حد زیادی به نحوه مدیریت کانال توسط آنها و سرویس دهنده ها بستگی دارد. مدیریت نادرست کانال ها توسط کاربران میتواند منجر به نامتعادل شدن کانال های آن ها شود و معمولا هزینه اصلاح یک کانال نامتعادل بسیار زیاد است. از طرف دیگر، مدیریت نادرست کانال ها توسط سرویس دهنده ها باعث کاهش سود آنها میشود. اگر سود ناشی از انتقال تراکنش ها در برابر هزینه های نگهداری کانال قابل توجه نباشد، سرویس دهنده ها انگیزه ای برای ماندن در شبکه کانال های پرداخت ندارند. این امر برای کارکرد شبکه کانال های پرداخت بسیار خطرناک است زیرا وجود واسطه ها برای انتقال تراکنش کاربرانی که کانال مستقیم ندارند الزامی است. در نتیجه ارائه الگوریتمی که این مسئله را با کمترین فروض و به صورت بهینه حل کند، بسیار ارزشمند است. 

\قسمت{دست آورد های تحقیق}

در این پایان نامه برای حل مسأله مدیریت آنلاین کانال های پرداخت، ابتدا از حل تئوری یک نسخه بسیار ساده شده و غیرواقع گرایانه مسأله شروع میکنیم و سپس در دو گام مدل را پیچیده تر واقع گرایانه تر میکنیم طوری که مسأله نهایی تا حد خوبی بیشتر پیچیدگی های کانال های پرداخت در دنیای واقعی را در خود دارد. این دو زیر مسأله به شرح زیر هستند:

\شروع{شمارش}
\فقره 
\مهم{زیر مسأله 1(کانال یکطرفه همیشه پذیرنده\پاورقی{Unidirectional stream without rejection})} در این زیر مسأله یک کانال پرداخت را در نظر میگیریم که منحصرا یکی از طرفین کانال  تراکنش میفرستد و جهت تراکنش ها همیشه یکطرفه است. همچنین فرض میکنیم تمام تراکنش ها باید پذیرفته شوند بنابراین هرگاه طرفین کانال موجودی کافی برای انتقال تراکنشی نداشتند، باید با  شارژ کردن درون بلاکچینی کانال خود را شارژ کنند تا بتوانند تمام تراکنش ها را عبور دهند. همچنین در این مدل متعادل کردن برون بلاکچینی مجاز نیست. برای این مدل الگوریتمی آنلاین با ضریب رقابتی \زیرنویس{\کد{competitive ratio} معیاری است که هزینه یک الگوریتم آنلاین را با هزینه الگوریتم بهینه آفلاین که از پیش به تمام تراکنش های آینده دسترسی دارد، مقایسه میکند. در قسمت \رجوع{قسمت:الگوریتم آنلاین} به طور مفصل این معیار و نحوه محاسبه آن را توضیح میدهیم.} 2 ارائه میدهیم و اثبات میکنیم که این بهترین ضریب رقابتی ای ست که یک الگوریتم آنلاین میتواند به آن دست یابد.

\فقره 
\مهم{زیر مسأله 2(کانال یکطرفه مجاز به رد تراکنش\پاورقی{Unidirectional stream with rejection})}
در این زیر مسأله همانند مدل قبلی جهت تراکنش ها همیشه یکطرفه است اما این بار دارندگان کانال میتوانند تصمیم بگیرند کدام تراکنش ها را انتقال دهند و کدام ها را رد کنند. پذیرفتن تراکنش ها پاداش کارمزد را به همراه دارد. مشابه مدل قبل
متعادل کردن برون بلاکچینی مجاز نیست. برای این مدل الگوریتم آنلاین با ضریب رقابتی
$2+\frac{\sqrt{5}-1}{2}$  
ارائه میدهیم و اثبات میکنیم که این الگوریتم بهینه است.
\فقره 
\مهم{مسأله اصلی(کانال دوطرفه \پاورقی{Bidirectional stream})}
در کلی ترین حالت مسأله تراکنش ها در هر دو جهت وجود دارند و صاحبان کانال نه تنها میتوانند تراکنش ها را به دلخواه بپذیرند یا رد کنند بلکه میتوانند از هر دو روش شارژ کردن درون بلاکچینی و متعادل کردن برون بلاکچینی برای متعادل کردن کانال خود استفاده کنند. برای این مدل الگوریتم آنلاین با ضریب رقابتی
 $7+\log{C}$
طراحی میکنیم.
(
$C$
یک عدد ثابت است که بستگی به ویژگی های گراف شبکه کانال های پرداخت دارد و مثلا در \کد{Lightning Network} حدودا برابر $4$ است
). 
همچنین به عنوان کران پایین نشان میدهیم که هیچ الگوریتم آنلاینی با ضریب رقابتی 
$o(\sqrt{\log{C}})$
 وجود ندارد.
\پایان{شمارش}

الگوریتم ها و اثبات های تئوری زیرمسأله 1 و 2 به عنوان بلوک های سازنده برای حل مسأله اصلی مورد استفاده قرار میگیرد. 

دقت کنید که در این بخش مسأله به نحوی بیان شد که همواره یکی از طرفین کانال پرداخت سرویس دهنده باشد اما در \رجوع{} توضیح میدهیم که چگونه باید پارامتر های مسأله را تغییر داد تا الگوریتم ها بتوانند برای مدیریت کانال دو کاربر عادی که قصد انتقال تراکنش سایرین را ندارند و فقط تراکنش های بین خودشان را میخواهند مدیریت کنند، مورد استفاده قرار گیرد.


\قسمت{ساختار پایان‌نامه}

این پایان‌نامه شامل پنج فصل است. 
فصل دوم دربرگیرنده‌ی تعاریف اولیه‌ی مرتبط با پایان‌نامه است. 
در فصل سوم مسئله‌ی دورهای ناهمگن و کارهای مرتبطی که در این زمینه انجام شده به تفصیل بیان می‌گردد. 
در فصل چهارم نتایج جدیدی که در این پایان‌نامه به دست آمده ارائه می‌گردد. در این فصل، مسئله‌ی درخت‌های ناهمگن در چهار شکل مختلف مورد بررسی قرار می‌گیرد. سپس نگاهی کوتاه به مسئله‌ی مسیرهای ناهمگن خواهیم داشت. در انتها با تغییر تابع هدف، به حل مسئله‌ی کمینه کردن حداکثر اندازه‌ی درخت‌ها می‌پردازیم.
فصل پنجم به نتیجه‌گیری و پیش‌نهادهایی برای کارهای آتی خواهد پرداخت.
