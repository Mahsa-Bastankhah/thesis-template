
\فصل{مقدمه}


بلاکچین هایی همچون بیتکوین به دلیل ماهیت توزیع شده والگوریتم اجماع پیچیده و وقت گیری که دارند با مشکل عدم مقیاس پذیری روبرو هستند. عدم مقیاس پذیری به این معنی است که سیستم نمیتواند تعداد بسیار زیاد تراکنش را پردازش کند. به طور مثال بلاکچین بیتکوین تنها میتواند 7 تراکنش در ثانیه را پردازش کند در حالیکه رقبای متمرکز بلاکچین همچون \کد{visa} بیش از هزاران تراکنش را در هر ثانیه پردازش میکنند. به علاوه، حتی وقتی تراکنش ها وارد بلاکچین میشوند تایید شدن آن ها معمولا حداقل چند دقیقه به طول می انجامد، به طور مثال در بلاکچین بیتکوین نزدیک یک ساعت طول میکشد تا یک تراکنش تأیید نهایی شود. یکی از مورد استقبال ترین راه حل هایی که برای حل مشکل مقیاس ناپذیری و کندی بلاکچین ارائه شده است استفاده از شبکه کانال های پرداخت\پاورقی{payment channel network} است.
شبکه کانال های پرداخت اولین بار با پیاده سازی \کد{Lightning Network} روی بلاکچین بیتکوین معرفی شد.
~\cite{poon2015lightning}
بعد ها شبکه کانال های پرداخت \کد{Raiden} هم با الهام از \کد{Lightning Network} بر بلاکچین اتریوم\پاورقی{Ethereum} توسعه داده شد.~\cite{raiden}
کاربران میتوانند با ارسال یک تراکنش ایجاد کانال \پاورقی{channel creation} روی بلاکچین، یک کانال پرداخت ایجاد کنند. با این تراکنش در واقع طرفین کانال پرداخت، مقداری پول را در این کانال پرداخت به سپرده میگذارند. پس از ایجاد کانال، طرفین میتوانند بدون مراجعه به بلاکچین و با رد و بدل کردن تعدادی امضای دیجیتال برای هم تراکنش محلی فوری\پاورقی{instant} با کارمزد بسیار اندک و بفرستند. مبادله امضاهای دیجیتال برای حفظ امنیت مالی طرفین الزامی است.

 شکل \رجوع{شکل:کانال پرداخت} نحوه کار یک کانال پرداخت را نشان میدهد. ابتدا کاربر \کد{َA} $2$ واحد پول و کاربر \کد{َB} $4$ واحد پول سپرده میکند و یک کانال پرداخت میسازند. تراکنش ایجاد کانال روی بلاکچین قرار میگیرد. پس از ایجاد کانال امکان ارسال تراکنش برون بلاکچینی فراهم میشود. \کد{َB} میخواهد برای \کد{َA} دو واحد پول واریز کند پس \کد{َA} و \کد{َB} امضاهای دیجیتال رد و بدل میکنند و دو واحد پول به صورت برون بلاکچینی به موجودی \کد{َA} اضافه میشود. پس از مدتی \کد{َA} تصمیم میگیرد کانال را ببندد؛ از امضاهای رد و بدل شده پیشین استفاده میکند تا یک تراکنش بستن کانال ایجاد کند. پس از اجرای این تراکنش هر کدام از \کد{َA} و \کد{َB} سه واحد پول میگیرند. توجه کنید که در مرحله دوم \کد{َA} و \کد{َB} میتوانند به تعداد نامحدود تراکنش برون بلاکچینی ایجاد کنند. پس با دو تراکنش درون بلاکینی، امکان ایجاد تعداد نامحدود ترامنش برون بلاکچینی ارزان و سریع فراهم شد. اما باید توجه کرد که مجموع موجودی \کد{َA} و \کد{َB} که به آن ظرفیت کانال \پاورقی{capacity} می گویند همواره عدد ثابت $6$ است و قابل افزایش یا کاهش نیست.  
 

\شروع{شکل}[hb]
\centerimg{paymentChannel}{10cm}
\شرح{نحوه شکل گیری، استفاده و بستن یک کانال پرداخت}
\برچسب{شکل:کانال پرداخت}
\پایان{شکل}



از اتصال کاربران مختلف با کانال های پرداخت یک شبکه از کانال های پرداخت ایجاد میشود که میتواند افرادی که کانال پرداخت مستقیم به هم ندارند را هم با یک یا تعدادی واسطه به هم متصل کند. مثلا 3 کاربر \کد{A-B-C} را در نظر بگیرید که \کد{A-B} و \کد{B-C} کانال پرداخت دارند. در این صورت \کد{َA} و \کد{َC} اگرچه کانال پرداخت مشترک ندارند اما میتوانند از \کد{َB} به عنوان واسطه استفاده کنند و برای هم تراکنش برون بلاکچینی ارسال کنند؛ بدین صورت که \کد{َA} مقدار پول مورد نظر را برای \کد{B} می فرستد و \کد{َB} همان مقدار پول را برای \کد{َC} میفرستد. این دو تراکنش ها \کد{atomic} هستند که به این معنی است که یا هر دو آنها انجام میشوند و یا هر دو برگشت میخورند. معمولا فرد واسطه یعنی \کد{َB} مقداری کارمزد از  \کد{َA} میگیرد اما این کارمزد در برابر کارمزد های تراکنش های درون بلاکچینی بسیار ناچیز است و صرفا نقش ایجاد انگیزه برای واسطه ها را دارد. البته گاهی نود های واسطه ممکن است برخی تراکنش ها را به دلایلی رد کنند. مثلا ممکن است اندازه تراکنش بیشتر از موجودی آن نود واسطه در کانال باشد یا اینکه موجودی نود واسطه را در حد غیر قابل قبولی کاهش دهد و یا اینکه میزان کارمزد آن مطلوب نود واسطه نباشد. 

یکی از مشکلات بسیار مهم شبکه کانال های پرداخت این است که بعد از ایجاد کانال هیچ راهی برای افزودن سپرده به کانال وجود ندارد. مثلا در مثال بالا در کانال \کد{A-B} فرض کنید با شروع از سپرده اولیه
 $2-4$
 ،\کد{A} 2 تراکنش هر کدام به ارزش 1 بیتکوین برای \کد{B} میفرستد؛ پس از انجام این دو تراکنش موجودی آنها در کانال به ترتیب
$0-6$   
خواهد بود. پس از این تا زمانی که \کد{B}  تراکنشی برای \کد{A} نفرستد، \کد{A} نمیتواند تراکنشی بفرستد زیرا موجودی اش صفر است. به کانال پرداختی که در آن موجودی یک نفر صفر(یا بسیار کم است) کانال نامتعادل \پاورقی{depleted channel} میگوییم. کانال های نامتعادل برای کاربران به خصوص برای نود های واسطه اصلا مطلوب نیستند زیرا امکان ایجاد تراکنش از یک سمت کانال را به کل از بین میبرند. در این پایان نامه گاهی نود های واسطه را سرویس دهنده \پاورقی{service provider} می نامیم. نود های سرویس دهنده نود هایی هستند که با هدف درآمد سازی به شبکه کانال های پرداخت ملحق میشوند و با ذخیره کردن مقدار چشم گیری سپرده، تعداد زیادی کانال با کاربران زیادی ایجاد میکنند تا تراکنش های آنها را مسیریابی کنند و در ازای آن کارمزد بگیرند. داشتن کانال های نامتعادل توانایی سرویس دهنده ها را در انتقال تراکنش ها از یک جهت کاهش میدهد و برای کسب و کار آنها مشکل ایجاد میکند. \کد{Lightning network} دو راه حل را برای حل مشکل کانال های نامتعادل پیشنهاد میدهد:


\برچسب{شمارش:راه های مقابله با کانال نامتعادل}\شروع{شمارش}

\فقره 
 شارژ کردن درون بلاکچینی: در این روش طرفین کانال نامتعادل آن کانال را میبندند و کانال جدیدی با سپرده بیشتر باز میکنند. این عمل باعث ایجاد دو تراکنش درون بلاکچینی میشود. یک تراکنش برای بستن کانال قدیمی و یک تراکنش برای ایجاد کانال جدید.
\فقره 
متعادل کردن برون بلاکچینی: این روش بدون مراجعه به بلاکچین و صرفا با تعدادی تراکنش برون بلاکچینی توزیع سپرده ها را در کانال نامتعادل تغییر میدهد و به نسبت روش قبل ارزان تر است. در بخش \رجوع{} به طور مفصل این روش را توضیح میدهیم. 
%این عمل به این صورت انجام میگیرد که طرفین کانال یک مسیر دوم از کانال های پرداخت بین خودشان پیدا میکنند و با جایجایی پول در یک حلقه فردی که همه پول را در کانال نامتعادل به دست دارد برای فرد دیگر اندکی پول میریزد و معادل آن پول را در یک کانال دیگر از آن نود دریافت میکند.
\پایان{شمارش}

 از آنجاییکه هر دو روش بالا هزینه بر هستند و محدودیت هایی را اعمال میکنند، تصمیم گیری بر سر اینکه چه زمانی کدام یک از آنها انجام گیرد تصمیم سختی است. همچنین توجه کنید که به عنوان یک کاربر یا یک نود واسطه، معمولا نود اطلاعات دقیقی از تراکنش های آینده ندارد و در نتیجه نود ها باید سیاست تصمیم گیری ای را اتخاذ کنند که بر اساس تاریخچه و بدون فرضی روی تراکنش های آینده، تصمیم گیری میکند.
 
 
هدف این پایان نامه این است که سیاست آنلاینی طراحی کند یک تک کانال پرداخت را در کلی ترین حالت ممکن در نظر 
میگیرد و به سوالات زیر که برای بیشینه کردن سود و کمینه کردن هزینه طرفین کانال مطرح میشود پاسخ میدهد:

\شروع{شمارش}

\فقره 
چه زمانی ایجاد یک کانال پرداخت نسبت به ارسال درون بلاکچینی تراکنش به مقرون به صرفه است؟
\فقره 
اگر تصمیم به ایجاد کانال پرداخت شد، طرفین چه مقدار سپرده باید در آن قرار دهند؟
\فقره 
اگر طرفین میخواهند نقش واسطه را ایفا کنند چه تراکنش هایی را باید بپذیرند و چه تراکنش هایی را نپذیرند؟
\فقره 
اگر کانال پرداخت نامتعادل شد، طرفین باید کدام یک از راه های  مقابله با کانال نامتعادل را اتخاذ کنند و سپرده کانال را چقدر باید تغییر دهند؟
\پایان{شمارش}


\قسمت{اهمیت موضوع}


هدف از طراحی شبکه کانال های پرداخت ایجاد بستری ارزان و سریع برای انجام تراکنش های کوچک و روزانه \پاورقی{micro payment} است. بهره بری کاربران از شبکه کانال های پرداخت تا حد زیادی به نحوه مدیریت کانال توسط آنها و سرویس دهنده ها بستگی دارد. مدیریت نادرست کانال ها توسط کاربران میتواند منجر به نامتعادل شدن کانال های آن ها شود و معمولا هزینه اصلاح یک کانال نامتعادل بسیار زیاد است. همچنین مدیریت نادرست کانال ها توسط سرویس دهنده هاهم به ضرر خود سرویس دهنده ها و هم به ضرر کاربران است. اگر سرویس دهنده ها نتوانند کانال های خود را درست مدیریت کنند، سود آنها کاهش می یابد و انگیزه ای برای ارائه خدمات نخواهند داشت که با توجه به اهمیت حیاتی سرویس دهنده ها برای شبکه، این امر بسیار مضر است. با بررسی آخرین داده های موجود از \کد{Lightning Network} ~\cite{lngossip} میتوان دید که در سال $2021$ حدود $6300$ در شبکه وجود دارد که بیش از $50$ درصد آن ها تنها از $10$ سرویس دهنده خدمات میگیرند. یعنی اگر $10$ سرویس دهنده اصلی \کد{Lightning Network} عملکرد مناسبی نداشته باشند، نیمی از شبکه مختل خواهد شد! در واقع بدون وجودسرویس دهنده ها، امکان ارسال تراکنش های با واسطه از بین میرود و همه کاربران مجبورند کانال های دو به دو با هم ایجاد کنند.

در نتیجه ارائه الگوریتمی که این مسئله مدیریت کانال را در یک مدل واقع بینانه، با کمترین فروض محدود کننده و به صورت بهینه حل کند، بسیار ارزشمند است. 

\قسمت{دست آورد های تحقیق}

در این پایان نامه برای حل مسأله مدیریت آنلاین کانال های پرداخت، ابتدا از حل تئوری یک نسخه بسیار ساده شده و غیرواقع گرایانه مسأله شروع میکنیم و سپس در دو گام مدل را پیچیده تر واقع گرایانه تر میکنیم طوری که مسأله نهایی تا حد خوبی بیشتر پیچیدگی های کانال های پرداخت در دنیای واقعی را در بر دارد. این دو زیر مسأله به شرح زیر هستند:

\شروع{شمارش}
\فقره 
\مهم{زیر مسأله 1(کانال یکطرفه همیشه پذیرنده\پاورقی{Unidirectional stream without rejection})} کانال پرداختی ساده و غیرواقع نگرانه ای با دو کاربر \کد{A} و \کد{B} را در نظر بگیرید که در آن همیشه فقط \کد{A} برای \کد{B} پول میفرستد، یعنی کانال یکطرفه است. همچنین فرض کنید که باید تمام تراکنش ها حتما انجام شود و کاربران امکان رد کردن تراکنش ها را ندارند(اگر \کد{A} یک کاربر عادی باشد رد تراکنش به این معنی است که \کد{A} تراکنشش را خارج از کانال پرداخت و از طرق دیگر انجام میدهد و اگر \کد{A} یک سرویس دهنده باشد رد کردن تراکنش به این معنی است که \کد{A} تصمیم میگیرد از کارمزد این تراکنش صرف نظر کند و این تراکنش را مسیریابی نکند). همچنین برای ساده سازی فرض کنید که اگر پول \کد{A} در کانال پرداخت تمام شد، باید کانال را ببندد و کانال جدید باز کند یا به عبارت دیگر تنها راه متعادل کردن کانال، شارژ کردن درون بلاکچینی است و متعادل کردن برون بلاکچینی برای ساده سازی مجاز نیست. این مدل، اولین و ساده ترین مدلی است که بررسی میکنیم و برای آن الگوریتمی آنلاین با نسبت رقابتی \زیرنویس{\کد{competitive ratio} معیاری است که هزینه یک الگوریتم آنلاین را با هزینه الگوریتم بهینه آفلاین که از پیش به تمام تراکنش های آینده دسترسی دارد، مقایسه میکند. در قسمت \رجوع{قسمت:الگوریتم آنلاین} به طور مفصل این معیار و نحوه محاسبه آن را توضیح میدهیم.}برابر $2$ ارائه میدهیم و اثبات میکنیم که این بهترین نسبت رقابتی ای ست که یک الگوریتم آنلاین میتواند به آن دست یابد.

\فقره 
\مهم{زیر مسأله 2(کانال یکطرفه مجاز به رد تراکنش\پاورقی{Unidirectional stream with rejection})}
در این زیر مسأله همانند مدل قبلی جهت تراکنش ها همیشه یکطرفه است اما این بار دارندگان کانال میتوانند تصمیم بگیرند کدام تراکنش ها را انتقال دهند و کدام ها را رد کنند. مشابه مدل قبل
متعادل کردن برون بلاکچینی مجاز نیست. برای این مدل الگوریتم آنلاین با نسبت رقابتی
$2+\frac{\sqrt{5}-1}{2}$  
ارائه میدهیم و اثبات میکنیم که این الگوریتم بهینه است.
\فقره 
\مهم{مسأله اصلی(کانال دوطرفه \پاورقی{Bidirectional stream})}
در کلی ترین حالت مسأله تراکنش ها در هر دو جهت وجود دارند و صاحبان کانال نه تنها میتوانند تراکنش ها را به دلخواه بپذیرند یا رد کنند بلکه میتوانند از هر دو روش شارژ کردن درون بلاکچینی و متعادل کردن برون بلاکچینی برای متعادل کردن کانال خود استفاده کنند. برای این مدل الگوریتم آنلاین با نسبت رقابتی
 $7+2\log{C}$
طراحی میکنیم.
(
$C$
یک عدد ثابت است که بستگی به ویژگی های گراف شبکه کانال های پرداخت دارد و مثلا در \کد{Lightning Network} حدودا برابر $4$ است
). 
همچنین به عنوان کران پایین نشان میدهیم که هیچ الگوریتم آنلاینی با  نسبت رقابتی 
$o(\sqrt{\log{C}})$
 وجود ندارد.
\پایان{شمارش}

الگوریتم ها و اثبات های تئوری زیرمسأله 1 و 2 به عنوان بلوک های سازنده برای حل مسأله اصلی مورد استفاده قرار میگیرد. 



\begin{table}[t]
\centering
\begin{latin}
\begin{tabular}{|c|c|c|}
\hline
\rl{کران پایین} & \rl{نسبت رقابتی} & \rl{مسئله‌}
\\
\hline
\hline
$2$ & $2$ & Unidirectional stream without rejection
 \\
$2+\frac{\sqrt{5}-1}{2}$  & $2+\frac{\sqrt{5}-1}{2}$  & Unidirectional stream with rejection
\\
$\theta(\sqrt{\log{C}})$  &  $7+2\log{C}$  &  Bidirectional stream
 \\
\hline
\end{tabular}
\end{latin}
\شرح{خلاصه نتایج تئوری این پایان نامه. ستون اول نام (زیر)مسأله، ستون دوم نسبت رقابتی و ستون}
\برچسب{جدول:نسبت های رقابتی}
\end{table}

%دقت کنید که در این بخش مسأله به نحوی بیان شد که همواره یکی از طرفین کانال پرداخت سرویس دهنده باشد اما در \رجوع{} توضیح میدهیم که چگونه باید پارامتر های مسأله را تغییر داد تا الگوریتم ها بتوانند برای مدیریت کانال دو کاربر عادی که قصد انتقال تراکنش سایرین را ندارند و فقط تراکنش های بین خودشان را میخواهند مدیریت کنند، مورد استفاده قرار گیرد.


\قسمت{ساختار پایان‌نامه}

این پایان‌نامه شامل پنج فصل است. 
فصل دوم دربرگیرنده‌ی تعاریف اولیه‌ی مرتبط با پایان‌نامه است. 
در فصل سوم مسئله‌ی دورهای ناهمگن و کارهای مرتبطی که در این زمینه انجام شده به تفصیل بیان می‌گردد. 
در فصل چهارم نتایج جدیدی که در این پایان‌نامه به دست آمده ارائه می‌گردد. در این فصل، مسئله‌ی درخت‌های ناهمگن در چهار شکل مختلف مورد بررسی قرار می‌گیرد. سپس نگاهی کوتاه به مسئله‌ی مسیرهای ناهمگن خواهیم داشت. در انتها با تغییر تابع هدف، به حل مسئله‌ی کمینه کردن حداکثر اندازه‌ی درخت‌ها می‌پردازیم.
فصل پنجم به نتیجه‌گیری و پیش‌نهادهایی برای کارهای آتی خواهد پرداخت.
