
\فصل{مفاهیم اولیه}
در این فصل مفاهیم اولیه لازم برای فهم مسأله و نتایج پایان نامه را مطرح میکنیم. ابتدا نحوه پردازش تراکنش ها در بلاکچین بیتکوین را توضیح میدهیم و سپس توضیح میدهیم که تراکنش های یک کانال پرداخت چه تفاوتی با تراکنش های عادی درون بلاکچینی دارند و چگونه میتوان تراکنش برون بلاکچینی امن داشت. در این بخش هنگام توضیح جزئیات پروتکل ها بلاکچین بیتکوین و شبکه کانال های پرداخت آن یعنی \کد{Lightning network} را به عنوان معیار در نظر میگیریم زیرا اولا امروزه \کد{Lightning network} با داشتن بیش از 17000 نود، پرکاربر ترین شبکه کانال های پرداخت موجود است 
~\cite{1ml}
و ثانیا مفاهیم پایه ای تراکنش های درون بلاکچینی و برون بلاکچینی کمابیش برای تمام بلاکچین ها یکسان است و تنها تفاوت در پروتکل های پیاده سازی شده است، پس تفاوت چندانی ندارد که کدام بلاکچین را به عنوان معیار قرار دهیم. 

\قسمت{تراکنش ها در بیتکوین}
بیتکوین یک بلاکچین \کد{UTXO-based} است، در این سیستم هر تراکنش یک یا تعدادی ورودی و یک یا تعدادی خروجی دارد. در هر تراکنش مجموع بیتکوین ورودی ها برابر است با مجموع بیتکوین خروجی ها و کارمزد تراکنش. شکل \رجوع{شکل:تراکنش} را ببینید. هر تراکنش یک هش \پاورقی{hash} یکتا دارد که شناسه تراکنش محسوب میشود. 
هر کدام از ورودی های یک تراکنش یکی از خروجی های یک تراکنش قدیمی تر را خرج میکند. هر خروجی دو داده در بر دارد : 1)مقدار پول موجود در آن 2) یک کد قفل کننده (\کد{ScriptPubKey}) که کلید عمومی \پاورقی{public key} و سایر مشخصات کسی که میتواند خروجی را خرج کند مشخص میکند. 
مثلا در شکل \رجوع{شکل:تراکنش} به خروجی شماره $1$ تراکنش سمت چپ دقت کنید. این خروجی دو بیتکوین دارد و  کد قفل کننده \کد{ScriptPubKey\_1}، کلید عمومی کسی که میتواند این پول را خرج کند مشخص میکند. این خروجی توسط ورودی شماره $1$ تراکنش سمت راست خرج میشود. 
هر ورودی سه داده را در بر دارد 1) هش تراکنشی که میخواهد یکی از خروجی های آن را خرج کند (در این مثال هش تراکنش سمت چپ که با قرمز رنگ مشخص شده است) 2) شماره خروجی مورد نظر( در این مثال، عدد $1$ که با سبز رنگ مشخص شده است ) 3) یک کد باز کننده قفل که شامل امضای صاحب پول و سایر اثبات های مورد نیاز  خروجی است (در این مثال، \کد{ScriptSig\_1} که با رنگ سرمه ای مشخص شده است) .  همچنین دقت کنید که در تراکنش سمت راست ورودی $2$ بیتکوین دارد و مجموع خروجی ها $1.99$ بیتکوین است؛ $0.01$ بیتکوین هم به عنوان کارمزد شبکه در نظر گرفته شده است.

 

\شروع{شکل}[hb]
\centerimg{tx}{11cm}
\شرح{ساختار یک تراکنش در بیتکوین}
\برچسب{شکل:تراکنش}
\پایان{شکل}


توجه کنید که کد قفل کننده و باز کننده قفل باید سازگار باشند مثلا دو عبارت زیر سازگار هستند:

\شروع{شکل}[hb]
\centerimg{lock1.png}{6cm}
\پایان{شکل}

کد قفل کننده میتواند شروط بیشتر و پیچیده تری هم برای خرج کننده خروجی ایجاب کند. مثلا به کد قفل کننده و بازکننده زیر توجه کنید که به امضای هر دو کاربر \کد{A}  و  \کد{B} نیازدارد. این خروجی مانند یک حساب دو کاربره عمل میکند زیرا برای خرج کردن پول آن تایید هر دو کاربر \کد{A}  و  \کد{B} لازم است.
%همچنین برای استفاده از این خروجی باید حداقل 5 دقیقه از تایید شدن این تراکنش گذشته باشد.

\شروع{شکل}[h]
\centerimg{lock2.png}{7cm}
\پایان{شکل}

\paragraph{قفل زمانی}
تراکنش های بیتکوین میتوانند شامل جزئیات دیگری مانند قفل زمانی\پاورقی{time lock} هم باشند. قفل زمانی به این معنی است که یک تراکنش را پیش از زمان مقرر نمیتوان به بلاکچین ارسال کرد. به طور مثال اگر شما تراکنشی با قفل زمانی \کد{December 31} بسازید و آن را پیش از این تاریخ به بلاکچین ارسال کنید، ماینر ها این تراکنش را ثبت نمیکنند و تا روز \کد{December 31} صبر کرده و بعد آن را ثبت میکنند.

\paragraph{خروجی های خرج نشده}
\پاورقی{UTXO}
به خروجی هایی که تاکنون خرج نشده اند \کد{Unspent Transaction Output (UTXO)} میگویند.  ماینر\پاورقی{miner} ها در شبکه بیتکوین دو وظیفه اصلی دارند:
\شروع{شمارش}
\فقره
 اطمینان حاصل کنند که هیچ خروجی ای بیش از یکبار خرج نمیشود .
\فقره
 بررسی کنند که هر کد باز کننده به درستی کد قفل کننده متناظر را باز میکند.
\پایان{شمارش}

\قسمت{تراکنش های کانال پرداخت}

در این بخش نوع ساخت و استفاده از کانال پرداخت های بر مبنای زمان\پاورقی{time based payment channels} ها را توضیح میدهیم. کانال پرداخت های رایج در \کد{Lightning Network} معمولا از نوع \کد{punishment based payment channel} هستند که  جزئیات پیچیده تری نسبت به کانال های بر مبنای زمان دارند. با این وجود چون اصول اولیه و کلیات هر دو این پروتکل ها مشابه هم است، فهم اصول کانال های بر مبنای زمان کافی است. برای خواندن درباره تفاوت این دو پروتکل ایجاد کانال میتوانید به منبع
~\cite{paymentChannBlog} 
مراجعه کنید.


\subsection{ایجاد کانال}
همانطور که پیش از این گفته شد، دو کاربر برای ایجاد کانال پرداخت باید یک تراکنش درون بلاکچینی "ایجاد کانال" بسازند. شکل \رجوع{شکل:تراکنش ایجاد کانال} تراکنش \کد{Tx1} را نشان میدهد که نمونه ای از یک تراکنش ایجاد کانال است. \کد{A} یکی از \کد{UTXO} هایش به ارزش $2$ بیتکوین و B یکی از \کد{UTXO} هایش به ارزش $4$ بیتکوین را در کانال سپرده میکنند. خروجی \کد{Tx1} یک \کد{UTXO} دو کاربره با موجودی 6 است. 

\شروع{شکل}[h]
\centerimg{channelCreation.png}{8cm}
\شرح{مثالی از یک تراکنش ایجاد کانال پرداخت. این تراکنش درون بلاکچینی است یعنی روی بلاکچین بیتکوین فرستاده میشود.}
\برچسب{شکل:تراکنش ایجاد کانال}
\پایان{شکل}
\subsection{استفاده از کانال}
همزمان یا اندکی پیش از امضای تراکنش درون بلاکچینی \کد{Tx1}، \کد{A}  و  \کد{B} مشترکا یک تراکنش برون بلاکچینی (\کد{Tx2}) هم ساخته و هر دو آن امضا میکنند اما آن را روی بلاکچین نمیفرستند، این تراکنش تنها در حافظه محلی \کد{A}  و  \کد{B} ذخیره میشود. 

\شروع{شکل}[h]
\centerimg{state0.png}{6.5cm}
\شرح{تراکنش حالت صفر کانال پرداخت. این تراکنش برون بلاکچینی است یعنی توسط \کد{A}  و  \کد{B} مشترکا ساخته و امضا شده و ذخیره می شود ولی تا زمان بسته شدن کانال روی بلاکچین قرار نمیگیرد.}
\برچسب{شکل:تراکنش حالت صفر}
\پایان{شکل}


\کد{Tx2} تک خروجی تراکنش \کد{Tx1} را خرج میکند و ، پول موجود در کانال را به همان نسبت اولیه $2-4$ بین \کد{A}  و  \کد{B} تقسیم میکند.
% برای استفاده از  خروجی \کد{Tx2} حداقل 30 روز از ثبت آن در بلاکچین باید گذشته باشد. 
تراکنش \کد{Tx2} تنها پس از زمان مقرر قفل زمانی(در این مثال \کد{December 31}) بر بلاکچین ثبت میشود.
\کد{Tx2} در واقع توزیع پول در کانال را در لحظه ایجاد آن یا لحظه صفر نشان میدهد به همین دلیل به آن تراکنش لحظه صفر میگوییم. در صورتی که هیچ تراکنشی بین \کد{A}  و  \کد{B}  انجام نگیرد و پس از مدتی یکی از \کد{A}  یا  \کد{B} تصمیم بگیرد کانال را ببندد، آن فرد میتواند تراکنش \کد{Tx2} را روی بلاکچین قرار دهد. چون \کد{Tx2}  پیش از این توسط هر دو \کد{A}  و  \کد{B} امضا شده است پس تراکنش معتبری است و بعد از تاریخ  قفل زمانی آن \کد{A}  و  \کد{B}  هر کدام میتوانند سهم خود را از کانال پرداخت بگیرند و کانال بسته میشود. اما در عمل \کد{A}  و  \کد{B}  کانال پرداخت ساخته اند تا از آن استفاده کنند نه اینکه آن را بلافاصله ببندند پس تراکنش \کد{Tx2} عملا هیچگاه استفاده نمیشود. این تراکنش تنها و تنها ساخته میشود تا به هر دو طرف تضمین دهد که اگر طرف دیگر پاسخگو نبود، پول آن ها در کانال قفل نمیماند و هر کدام میتوانند سهم خود را از کانال دریافت کنند.

اکنون با یک مثال توضیح میدهیم که در عمل چگونه از کانال پرداخت ایجاد شده در شکل \رجوع{شکل:تراکنش ایجاد کانال} استفاده میشود. فرض کنید  \کد{B} میخواهد برای \کد{A} 1 بیتکوین در کانال پول بریزد. \کد{B} ترکنش برون بلاکچینی \کد{Tx3} نمایش داده شده در شکل \رجوع{شکل:تراکنش حالت یک} را میسازد و امضا میکند و برای \کد{A} میفرستد. \کد{A} هم \کد{Tx3} را امضا کرده و ذخیره میکند. \کد{Tx3} توزیع پول موجود در کانال را به $3-3$ تغییر میدهد. هرگاه هر کدام از  \کد{A}  یا  \کد{B} بخواهند این کانال را ببندند کافی است \کد{Tx3} را روی بلاکچین بفرستد و در تاریخ \کد{December 30} سهم خود از کانال را پس بگیرند. 

\شروع{شکل}[h]
\centerimg{state1.png}{6.5cm}
\شرح{}
\برچسب{شکل:تراکنش حالت یک}
\پایان{شکل}

اما یک مشکل وجود دارد؛ چگونه تضمین دهیم که هیچ کدام از طرفین تراکنش قدیمی تر \کد{Tx2} را روی بلاکچین قرار نمیدهند تا کانال را با یک توزیع پول قدیمی ببندند؟ این عمل به طور خاص به نفع \کد{A} است زیرا موجودی \کد{A} در \کد{Tx2} بیشتر از \کد{Tx3} است. محدودیت زمانی اعمال شده روی خروجی تراکنش های \کد{Tx2} و \کد{Tx3} نقش جلوگیری از این عمل را دارد؛ فرض کنید \کد{A} میخواهد سر \کد{B} کلاه بگذارد و بدون هیچ اطلاع قبلی تراکنش \کد{Tx2} را روی بلاکچین قرار میدهد؛ \کد{B} با دیدن این عمل، تراکنش  \کد{Tx3}  را به بلاکچین میفرستد. قفل زمانی  \کد{Tx3} روز \کد{December 30} باز میشود پس این تراکنش یک روز زودتر از \کد{Tx2} قابل اجرا است. تراکنش  \کد{Tx3} در روز \کد{December 30} انجام میشود و بعد از آن، تراکنش \کد{Tx2} دیگر قابل اجرا نخواهد بود زیرا ورودی آن قبلا توسط  \کد{Tx3} مصرف شده است.



\قسمت{تراکنش های با واسطه}

\قسمت{\کد{Lightning Network}}

\قسمت{الگوریتم های آنلاین}\پاورقی{online algorithms}
