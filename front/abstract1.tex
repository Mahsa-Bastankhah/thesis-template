
% -------------------------------------------------------
%  Abstract
% -------------------------------------------------------


\pagestyle{empty}

\شروع{وسط‌چین}
\مهم{چکیده}
\پایان{وسط‌چین}
\بدون‌تورفتگی

امروزه استفاده از شبکه کانال های پرداخت\پاورقی{payment channel} مبتنی بر بلاکچین به عنوان یکی از عملی ترین راه حل های مشکل عدم مقیاس پذیری بلاکچین بسیار مورد توجه قرار گرفته است.  کاربران با استفاده از شبکه کانال های پرداخت برای تراکنش های روزمره،در عین اینکه از تمام تضمین های امنیتی و محرمانگی بلاکچین بهره مند میشوند، میتوانند از پرداخت کارمزد های سرسام آور بلاکچین خودداری کنند.  


نحوه کار کانال پرداخت به صورت خلاصه به شرح زیر است. دو نفر برای ایجاد یک کانال پرداخت باید با فرستادن تراکنش مخصوصی به بلاکچین، مقداری سپرده برای کانال خود ذخیره کنند. بعد از ایجاد کانال دو نفر میتوانند تا سقف سپرده خود تراکنش برون بلاکچینی \زیرنویس{بدون مراجعه به بلاکچین \کد{off-chain transaction} } برای هم بفرستند و چون این تراکنش ها به صورت محلی\پاورقی{local} و بدون مراجعه به بلاکچین انجام میشوند بسیار سریع هستند و کارمزد آن ها ناچیز است. در انتها وقتی طرفین تصمیم به بستن کانال خود میگیرند با ارسال تراکنش دیگری به بلاکچین میتوانند سپرده خود را آزاد کنند. بدین ترتیب با تنها دو تراکنش درون بلاکچینی\پاورقی{on-chain} یکی برای ایجاد کانال و دیگری برای بستن کانال، امکان ارسال صدها تراکنش برون بلاکچینی فراهم میشود.

یکی از مهم ترین محدودیت های کانال پرداخت این است که افراد امکان اضافه کردن سپرده به کانال را فقط و فقط در هنگام ایجاد کانال دارند و اگر بعدا تصمیم به افزایش سپرده خود بگیرند باید کانال را بسته و کانال جدیدی ایجاد کنند که امری هزینه بر است. بنابراین کاربران  تمایل دارند مقدار سپرده کافی در کانال از همان ابتدا قرار دهند اما از طرف دیگر نباید بیش از اندازه هم در کانال پول بگذارند زیرا امکان استفاده از این پول را تا بستن کانال نخواهند داشت. در نتیجه کاربران هنگام ایجاد کانال با یک مسأله تصمیم گیری آنلاین روبرو هستند. اما در عمل مساله از این هم پیچیده تر است زیرا میلیون ها کاربر با کانال پرداخت های دو به دویی که تشکیل میدهند، شبکه عظیمی از کانال های پرداخت \پاورقی{payment channel networks}  را تشکیل میدهند. در این شبکه هر دو نودی که که یک کانال پرداخت مشترک دارند میتوانند بی واسطه برای هم تراکنش بفرستند اما نود هایی که کانال مستقیم با هم ندارند باید با استفاده از سایر نود های شبکه به عنوان واسطه، تراکنش خود را در شبکه مسیریابی و ارسال کنند. به طور مثال شبکه ای به صورت \کد{A-B-C} را در نظر بگیرید که در آن نود های \کد{A} و \کد{C}کانال پرداخت مشترک ندارند اما هر دو با \کد{B} کانال مشترک دارند؛ در این شبکه \کد{A} میتواند برای \کد{B} پول بفرستد و \کد{B} همان پول را به \کد{C} ارسال کند و تراکنش \کد{A} به \کد{C} با یک واسطه انجام خواهد شد. بنابراین نود های موجود در شبکه میتوانند در دو نقش کاربر(فرستنده یا گیرنده) یا سرویس دهنده(واسطه) ایفای نقش کنند. نود های واسطه در ازای انتقال تراکنش های کاربران کارمزد دریافت میکنند پس تمایل دارند که تا حد امکان تراکنش های بیشتری را مسیریابی کنند؛ اما از طرفی اگر واسطه ها حریصانه تمام تراکنش های کاربران را مسیریابی کنند، کانال هایشان خالی از پول میشود. مثلا در مثال بالا فرض کنید \کد{A} قصد دارد تعداد تراکنش زیادی برای \کد{C} بفرستد، اگر \کد{B} تمام این تراکنش ها را مسیریابی کند، هیچ پولی در کانال \کد{B-C}  برای او باقی نخواهد بود و در عوض \کد{B}مقدار زیادی پول در کانال \کد{A-B} مقدار زیادی پول خواهد داشت. در چنین شرایطی میگوییم کانال \کد{B} نامتعادل شده است. نامتعادل شدن کانال امر مطلوبی نیست زیرا مانع انتقال تراکنش های بعدی در جهت خالی شده از پول میشود. پس نود های واسطه شبکه کانال های پرداخت در هر لحظه با یک تصمیم گیری آنلاین روبرو هستند؛ اینکه کدام یک از تراکنش های کاربران را انتقال دهند. بنابراین در مجموع میبینیم که نود های شبکه کانال های پرداخت چه هنگام ایجاد کانال و چه بعدا د زمان ارسال تراکنش های برون بلاکچینی باید مدام تصمیمات آنلاینی در خصوص مدیریت کانال خود بگیرند.

نود های شبکه کانال های پرداخت نیاز به الگوریتمی برای مدیریت کانال خود دارند. این الگوریتم باید آنی باشد به این معنی که الگوریتم برای اتخاذ تصمیمات زمان زیادی ندارد. طراحی یک الگوریتم بهینه آنلاین که درباره مدیریت سپرده های نود ها تصمیم گیری میکند نه تنها در این حوزه مورد نیاز است بلکه میتواند در حوزه های دیگر همچون شبکه های مخابراتی برای حل مسأله \کد{admission control} هم سود بخش باشد. در این پایان نامه الگوریتم آنلاینی برای مدیریت یک تک کانال پرداخت طراحی میکنیم. الگوریتم ما یک الگوریتم آنلاین است به این معنی که هیچ فرض خاصی روی توزیع تراکنش های آینده ندارند و تنها با اطلاعات گذشته و لحظه حال تصمیمی اتخاذ میکند. در این پایان کران بالای هزینه الگوریتممان را برای بدترین دنباله تراکنش \پاورقی{worst-case analysis} ممکن اثات میکنیم؛ و در نهایت با پیاده سازی نشان میدهیم که الگوریتم ما در عمل بسیار بهتر از تضمین تئوری اثبات شده عمل میکند و همچنین دو \کد{heuristic} با الهام از الگوریتم اصلی طراحی میکنیم که در عمل هزینه را تا نصف هزینه الگوریتم اصلی کاهش میدهد.
 
\پرش‌بلند
\بدون‌تورفتگی \مهم{کلیدواژه‌ها}: 
بلاکچین، شبکه کانال های پرداخت، الگوریتم آنلاین، \کد{admission control}
\صفحه‌جدید
