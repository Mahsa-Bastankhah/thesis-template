
% -------------------------------------------------------
%  Abstract
% -------------------------------------------------------


\pagestyle{empty}

\شروع{وسط‌چین}
\مهم{چکیده}
\پایان{وسط‌چین}
\بدون‌تورفتگی

امروزه استفاده از شبکه کانال های پرداخت\پاورقی{payment channel} مبتنی بر بلاکچین به عنوان یکی از عملی ترین راه حل های مشکل عدم مقیاس پذیری بلاکچین بسیار مورد توجه قرار گرفته است.  کاربران با استفاده از شبکه کانال های پرداخت برای تراکنش های روزمره،در عین اینکه از تمام تضمین های امنیتی و محرمانگی بلاکچین بهره مند میشوند، میتوانند از پرداخت کارمزد های سرسام آور بلاکچین خودداری کنند.  


نحوه کار کانال پرداخت به صورت خلاصه به شرح زیر است. دو نفر برای ایجاد یک کانال پرداخت باید با فرستادن تراکنش مخصوصی به بلاکچین، مقداری سپرده برای کانال خود ذخیره کنند. بعد از ایجاد کانال دو نفر میتوانند تا سقف سپرده خود تراکنش برای هم بفرستند و چون این تراکنش ها به صورت محلی\پاورقی{local} و برون بلاکچینی \زیرنویس{بدون مراجعه به بلاکچین \کد{off-chain transaction} }انجام میشود  بسیار سریع هستند و کارمزد آن ها ناچیز است. در انتها وقتی طرفین تصمیم به بستن کانال خود میگیرند با ارسال تراکنش دیگری به بلاکچین میتوانند سپرده خود را آزاد کنند. 

یکی از مهم ترین محدودیت های کانال پرداخت این است که افراد امکان اضافه کردن سپرده به کانال را فقط و فقط در هنگام ایجاد کانال دارند و اگر بعدا تصمیم به افزایش سپرده خود بگیرند باید کانال را بسته و کانال جدیدی ایجاد کنند که امری هزینه بر است. بنابراین  کاربران تمایل دارند تا حد امکان مقدار سپرده کافی در کانال خود از همان ابتدا قرار دهند. از طرفی کاربران مایل نیستند بیش از مقدار نیاز واقعی خود در کانال پول ذخیره کنند زیرا امکان استفاده از این پول را تا بستن کانال نخواهند داشت. در نتیجه کاربران هنگام ایجاد کانال با یک مسأله تصمیم گیری آنلاین روبرو هستند. اما وقتی شبکه ای از این کانال های پرداخت ایجاد میشود، تصمیمات متعدد دیگری هنگام نگهداری یک کانال باید گرفته شود.

یک کاربر میتواند به تعداد دلخواه با کاربران متفاوت دیگر، کانال پرداخت ایجاد کند و به این صورت از اتصال کاربران به هم توسط کانال های پرداخت، شبکه ای از کانال های پرداخت \پاورقی{payment channel network} ایجاد میشود. در این شبکه هر دو نودی که که یک کانال پرداخت مشترک دارند میتوانند بی واسطه برای هم تراکنش بفرستند اما نود هایی که کانال مستقیم با هم ندارند میتوانند در صورت امکان با استفاده از سایر نود های شبکه به عنوان واسطه، تراکنش خود را در شبکه مسیریابی و ارسال کنند. بنابراین نود های موجود در شبکه میتوانند در دو نقش کاربر(فرستنده یا گیرنده) یا سرویس دهنده(واسطه) ایفای نقش کنند. نود های واسطه با انتقال تراکنش های کاربران کارمزد دریافت میکنند اما از طرفی قبول حریصانه تمام تراکنش ها توسط واسطه ها ممکن است کانال های آنها را نامتعادل کند، به این معنی که موجودی یک طرف کانال صفر شود و همه سپرده دست طرف دیگر باشد که این امر مانع انتقال تراکنش های بعدی از جهت بدون پول میشود. بدین صورت نود های شبکه کانال های پرداخت با یک تصمیم گیری آنلاین دیگر هم روبرو هستند؛ اینکه کدام یک تراکنش های کاربران را انتقال دهند . بنابراین همانطور که توضیح داده شد میبینیم که نود های شبکه کانال های پرداخت چه هنگام ایجاد کانال و چه بعدا باید مدام تصمیمات آنلاینی در خصوص نگهداری کانال خود بگیرند.

طراحی یک الگوریتم بهینه آنلاین که درباره مدیریت سپرده های نو ها تصمیم گیری میکند نه تنها در این حوزه مورد نیاز است بلکه میتواند در حوزه های دیگر همچون شبکه های مخابراتی برای حل مسأله \کد{admission control}  هم سود بخش باشد. در این پایان نامه الگوریتم آنلاینی برای این مسأله تصمیم گیری به همراه تضمین تئوری کارکرد آن ارائه میدهیم. همچنین در نهایت با پیاده سازی نشان میدهیم که الگوریتم ما در عمل بسیار بهتر از تضمین تئوری اثبات شده عمل میکند و همچنین دو \کد{heuristic} با الهام از الگوریتم اصلی طراحی میکنیم که در عمل هزینه را تا نصف هزینه الگوریتم اصلی کاهش میدهد.
 
\پرش‌بلند
\بدون‌تورفتگی \مهم{کلیدواژه‌ها}: 
بلاکچین، شبکه کانال های پرداخت، الگوریتم آنلاین، \کد{admission control}
\صفحه‌جدید
